%% LyX 2.2.2 created this file.  For more info, see http://www.lyx.org/.
%% Do not edit unless you really know what you are doing.
\documentclass[british,english,showpacs,preprintnumbers,amsmath,amssymb,aps,notitlepage,twocolumn]{revtex4-1}
\usepackage[T1]{fontenc}
\usepackage[latin9]{inputenc}
\setcounter{secnumdepth}{3}
\usepackage{color}
\usepackage{array}
\usepackage{float}
\usepackage{units}
\usepackage{textcomp}
\usepackage{multirow}
\usepackage{amsmath}

\makeatletter

%%%%%%%%%%%%%%%%%%%%%%%%%%%%%% LyX specific LaTeX commands.
%% Because html converters don't know tabularnewline
\providecommand{\tabularnewline}{\\}

%%%%%%%%%%%%%%%%%%%%%%%%%%%%%% User specified LaTeX commands.
\usepackage{amsfonts}
%\usepackage{palatino}

\makeatother

\usepackage{babel}
\begin{document}

\title{\textcolor{black}{\normalsize{}Computer-predicted excitation energy
of carbon in agreement with experiment to within $1$~cm$^{-1}$}}
\selectlanguage{british}%

\author{Nike Dattani}
\email{dattani.nike@hertford.ox.ac.uk}

\selectlanguage{british}%

\affiliation{Oxford University, Hertford College, OX1 3BW, Oxford, UK,}

\affiliation{Kyoto University, Fukui Institute for Fundamental Chemistry, 606-8103,
Kyoto, Japan,}

\author{David Feller}
\email{dfeller@owt.com }

\selectlanguage{british}%

\affiliation{Department of Chemistry, Washington State University, Pullman, Washington
99164-4630, USA,}

\author{Jacek Koput}
\email{koput@amu.edu.pl}

\selectlanguage{british}%

\affiliation{Department of Chemistry, Adam Mickiewicz University, 60\textendash 780
Poznan, Poland.}
\selectlanguage{english}%

\date{\today}
\begin{abstract}
We show that with non-explicitly correlated basis sets, we can predict
the first excitation energy in carbon to within 1~cm$^{-1}$ of the
experimental value.
\end{abstract}
\selectlanguage{british}%

\author{}

\author{}
\maketitle
\selectlanguage{english}%

\section{Introduction}

In the last three years, excitation energies have been calculated
with unprecedented precision for the Li atom \cite{Puchalski2013},
Be atom \cite{Puchalski2013b} and B atom \cite{Puchalski2015}. Tight
variational bounds for non-relativistic ground state energies assuming
a clamped, point-sized nucleus have reached 49 digits in units of
Hartree for He, 19 digits for Li, 12 digits for Be, 11 digits for
B, and 6 digits for C (see Table \ref{tab:variational bounds}). Calculated
excitation energies have been made in agreement with experiment to
within $10^{-3}$~cm$^{-1}$ for $^{7}$Li, $10^{-1}$~cm$^{-1}$
for $^{9}$Be and $1$~cm$^{-1}$ for $^{10}$B (see Table \ref{tab:excitationEnergies}). 

For the C atom, before this present study, no high-precision calculation
had been reported to predict an excitation energy to within $1$~cm$^{-1}$
agreement with experiment. Table \ref{tab:variational bounds} shows
that the method used for the calculation has a crucial impact on the
precision obtainable: the best variational bound on the non-relativistic,
clamped, point-nucleus (NR,CPN), ground state energy of Li is believed
to be 8 orders of magnitude more accurate than that of the Be atom,
because analytic expressions for the multi-center integrals of Hylleraas
wavefunctions are only known for up to 3e$^{-}$, and it would be
very slow to calculate them numerically for 4e$^{-}$. For 4e$^{-}$
and greater, ECGs (\textbf{e}xplicitly \textbf{c}orrelated \textbf{g}aussian
wavefunction\textbf{s)} are used. ECGs mimic the theoretical shape
of the wavefunction less accurately (e.g. they are rounded rather
than cusp-like in the limit of vanishing electron-electron distance,
and in the infinite electron-electron distance limit they decay as
$\nicefrac{1}{r^{2}}$ rather than $\nicefrac{1}{r}$), so more parameters
are needed than when using Hylleraas wavefunctions. However, the integrals
of ECGs are known analytically, regardless of the number of electrons,
which is why it has been possible to do very high-precision calculations
on Be \cite{Puchalski2013b} and B \cite{Puchalski2015}. 

Table \ref{tab:variational bounds} demonstrates though, that with
twice as many variationally optimizable parameters, the authors obtained
convergence in 1 fewer digit for the B atom than for the Be atom.
This suggests that it would be very difficult to use the ECG approach
for atoms and molecules coming from most of the periodic table. The
best known variational bound for the NR,CPN ground state energy for
C was calculated in 2015 using fixed-node diffusion Monte Carlo (FN-DMC)
with the nodes of the electronic wavefunction fixed at the locations
of the CISD/cc-pV5Z wavefunction, and the statistical uncertainty
on the Monte Carlo was $\pm2$~$E_{h}$. However, FN-DMC has not
yet been able to calculate the energies of excitated states. In this
paper, the approach we use to calculate the NR,CPN energy of the ground
state of C is FCIQMC (full configuration interaction Monte Carlo)
in an aug-cc-pCV7Z basis set. Fig \ref{tab:variational bounds} shows
that our NR,CPN energy is at least 210~$\mu E_{h}$ higher than the
variational upper bound obtained from FN-DMC; but since it is straightforward
to calcualte excited state energies with FCIQMC, unlike FN-DMC, we
are easily able to calculate the first excitation energy for carbon
$(2^{2}P\leftarrow2^{1}D)$ with high precision in this work. 

After adding estimates for the correction to the non-relativistic
approximation, and for the correction to the clamped nucleus approximation,
we obtained a $(2^{2}P\leftarrow2^{1}D)$ excitation energy for C
which is within 1~cm$^{-1}$ disagreement with the best known experimental
estimate for this transition. While this is not as impressive as the
method of variationally optimizing parameters in an explicitly correlated
wavefunciton ansatz has proven to be for Li and Be, the disagreement
has the same order of magnitude as the latter approach for B (see
Table \ref{tab:excitationEnergies}). We finally note that the approach
used in this paper, of calculating FCIQMC on a basis set of non-explicitly
correlated orbitals has successfully treated systems with far more
electrons (transition metal atoms \cite{Thomas2015a}, diatomics \cite{Cleland2012},
larger molecules such as butadiene \cite{Daday2012}, and even solid
state systems \cite{Booth2012a}), so it is conceivable that the approach
used in this paper may in the near future be able to determine (with
good accuracy) the first excitation energy for astatine, which at
present remains experimentally elusive. 
\begin{center}
\begin{table}[H]
\caption{\label{tab:finalFCIQMC}Final NR-CPN energies. The break-down of how
these energies were obtained, and the Hartree-Fock energies that were
used for the extrapolations are available in the Supplemental Material.
Numbers in parenthesis indicate uncertainties in the last digit(s)
shown, and their determination is described in the Supplemental Material.}

\begin{tabular*}{1\columnwidth}{@{\extracolsep{\fill}}llr@{\extracolsep{0pt}.}lr@{\extracolsep{0pt}.}lr@{\extracolsep{0pt}.}l}
\hline 
\noalign{\vskip2mm}
 &  & \multicolumn{2}{c}{C$\left(^{3}P\right)$ $\left[E_{{\rm Hartree}}\right]$} & \multicolumn{2}{c}{C$^{+}$$\left(^{2}P\right)\left[E_{{\rm Hartree}}\right]$} & \multicolumn{2}{c}{$2^{3}P\rightarrow2^{2}P$~{[}cm$^{-1}${]}}\tabularnewline[2mm]
\hline 
\hline 
\noalign{\vskip2mm}
aug-cc-pCV7Z &  & -37&844~251~5(005) & -37&430~345~1(001) & 90~841&955(020)\tabularnewline
\multirow{1}{*}{aug-cc-pCV8Z} &  & -37&844~355~5(008) & -37&430~412~5(005) & 90~849&987(050)\tabularnewline[2mm]
\hline 
\noalign{\vskip2mm}
CBS$\left(1/X^{3.5}\right)$ &  & -37&844~528~6 & -37&430~523~6 & 90~863&604\tabularnewline
CBS$\left(1/\left(X+\nicefrac{1}{2}\right)^{4}\right)$ &  & -37&844~514~2 & -37&430~514~3 & 90~862&471\tabularnewline[2mm]
\hline 
\noalign{\vskip2mm}
Mean ($\sqrt{\text{Variance}}$) &  & -37&844~521~4 & -37&843~051~9 & 90~863&037\tabularnewline[2mm]
\hline 
\end{tabular*}
\end{table}
\par\end{center}

\begin{center}
\begin{table}[H]
\caption{\label{tab:Summary}Summary of our main result.}

\begin{tabular*}{1\columnwidth}{@{\extracolsep{\fill}}llr@{\extracolsep{0pt}.}lr@{\extracolsep{0pt}.}l}
\hline 
\noalign{\vskip2mm}
 & Hamiltonian & \multicolumn{2}{c}{Energy in cm$^{-1}$} & \multicolumn{2}{c}{\textcolor{black}{\scriptsize{}(Obs - Calc) in cm$^{-1}$}}\tabularnewline[2mm]
\hline 
\hline 
\noalign{\vskip2mm}
$m\alpha^{2}$ & NR-CPN & 90~863&324(810) & \multicolumn{2}{c}{}\tabularnewline
\multirow{2}{*}{$m\alpha^{4},m\alpha^{5}$} & X2C & -30&023(050) & \multicolumn{2}{c}{}\tabularnewline
 & Breit \& QED & -0&5 & \multicolumn{2}{c}{}\tabularnewline
$\left(\frac{m^{2}}{m+m_{N}}\right)\alpha^{2}$ & DBOC & -0&5 & \multicolumn{2}{c}{}\tabularnewline
\hline 
\noalign{\vskip2mm}
Present  & Total theory & \textcolor{red}{90~832}&\textcolor{red}{301(810)} & 0&707\tabularnewline[2mm]
\hline 
\noalign{\vskip2mm}
2017  & Experiment & 90~833&008(009)  & \multicolumn{2}{c}{}\tabularnewline[2mm]
\hline 
\noalign{\vskip2mm}
2010 & Theory & 90~838&75 & -5&74\tabularnewline
2017 & Theory & 90~840&16 & -7&15\tabularnewline
2015 & Theory & 90~786&66  & 46&35\tabularnewline[2mm]
\hline 
\end{tabular*}
\end{table}
\par\end{center}

\begin{center}
\begin{table*}
\caption{\label{tab:variational bounds}Tight variational upper bounds for
atomic energies. For H, the listed energy is exact, since the Hartree
energy unit is defined as twice this energy, and can be related to
fundamental SI units through an analytic solution to the relavent
Schroedinger equation. VO stands for variational optimization (parameters
in a wavefunction ansatz are optimized in attempt to obtain the lowest
possible energy). Hylleraas-Log indicates the use of Hylleraas functions
supplemented with auxiliary log functions, and ECG($N$) indicates
the use of \textbf{e}xplicitly \textbf{c}orrelated \textbf{g}aussians
with $N$ variationally optimizable parameters. FN-DMC stands for
fixed node diffusion Monte Carlo.}

\begin{tabular*}{1\textwidth}{@{\extracolsep{\fill}}ccr@{\extracolsep{0pt}.}lcccccc}
\hline 
\noalign{\vskip2mm}
 &  & \multicolumn{2}{c}{Total non-relativistic clamped point-nucleus (NR,CPN) energy {[}Hartree{]}} & Method/Ansatz type & Reference &  &  &  & \tabularnewline[2mm]
\hline 
\hline 
\noalign{\vskip2mm}
H & 1 & -0&5 & Analytic & 1926 Schroedinger &  &  &  & \tabularnewline
He & 2 & -2&903 724 377 034 119 598 311 159 245 194 404 446 696 925 309 838 & VO/Hylleraas-Log & 2006 Schwartz &  &  &  & \tabularnewline
Li & 3 & -7&478 060 323 910 134 843 & VO/Hylleraas & 2011 Wang &  &  &  & \tabularnewline
Be & 4 & -14&667 356 494 9 & VO/ECG(4096) & 2013 Puchalski & \cite{Puchalski2013b} &  &  & \tabularnewline
B & 5 & -24&653 867 537 & VO/ECG(8192) & 2015 Puchalski & \cite{Puchalski2015} &  &  & \tabularnewline
C & 6 & -37&844 48(2) & FN-DMC & 2015 Yang &  &  &  & \tabularnewline
C & 6 & \textcolor{blue}{-37}&\textcolor{blue}{844~251~5(08)} & FCIQMC/aCV7Z & Present work & - &  &  & \tabularnewline
C & 6 & \textminus 37&843 333 & VO/ECG(1000) & 2013 Bubin & \cite{Bubin2013} &  &  & \tabularnewline[2mm]
\hline 
\end{tabular*}
\end{table*}
\par\end{center}

\begin{center}
\begin{table*}
\caption{\label{tab:excitationEnergies}The most precisely calculated electronic
excitation energies for the first 6 atoms, compared to the best known
experimental measurements to date. The last column indicates that
if aiming for the best precision, an experimental measurement is still
the best way to obtain the energy for most atoms, but for Be, the
energy has been obtained more precisely $\textit{in silico}$ than
in any experiment to date.}

\begin{tabular*}{1\textwidth}{@{\extracolsep{\fill}}lcr@{\extracolsep{0pt}.}lr@{\extracolsep{0pt}.}lr@{\extracolsep{0pt}.}lcr@{\extracolsep{0pt}.}lr@{\extracolsep{0pt}.}lc}
\hline 
\noalign{\vskip2mm}
 & \multirow{2}{*}{{\small{}Transition}} & \multicolumn{2}{c}{{\small{}Experiment }} & \multicolumn{2}{c}{} & \multicolumn{2}{c}{{\small{}Theory}} &  & \multicolumn{2}{c}{{\small{}Calc - Obs}} & \multicolumn{2}{c}{\multirow{2}{*}{{\small{}$\lvert\frac{\text{Calc - Obs}}{\text{Uncertainty in obs}}\rvert$}}} & \multirow{2}{*}{{\small{}More precise}}\tabularnewline
 &  & \multicolumn{2}{c}{{\small{}{[}cm$^{-1}${]}}} & \multicolumn{2}{c}{} & \multicolumn{2}{c}{{\small{}{[}cm$^{-1}${]}}} &  & \multicolumn{2}{c}{{\small{}{[}cm$^{-1}${]}}}\multicolumn{2}{c}{\multirow{2}{*}{{\small{}$\lvert\frac{\text{Calc - Obs}}{\text{Uncertainty in obs}}\rvert$}}} &  & \tabularnewline[2mm]
\hline 
\hline 
\noalign{\vskip2mm}
{\small{}$^{1}$H} & {\small{}H$^{+}\left(1^{1}S\right)\leftarrow\text{H}\left(1^{2}S\right)$} & & & \multicolumn{2}{c}{} & {\small{}109~678}&{\small{}771~743~07(10)} &  & \multicolumn{2}{c}{} & \multicolumn{2}{c}{} & {\small{}Theory}\tabularnewline
{\small{}$^{4}$He} & {\small{}He$^{+}\left(1^{2}S\right)\leftarrow\text{He}\left(1^{1}S\right)$} & {\small{}198~310}&{\small{}666~37(2)} & \multicolumn{2}{c}{} & {\small{}198~310}&{\small{}665~07(1)} &  & {\small{}-0}&{\small{}001~3} & {\small{}65}&{\small{}0} & {\small{}Theory}\tabularnewline
{\small{}$^{3}$Li} & {\small{}Li$^{+}\left(1^{1}S\right)\leftarrow\text{Li}\left(2^{2}S\right)$} & {\small{}3~487}&{\small{}159~40(18)} & \multicolumn{2}{c}{{\small{}\cite{Brown2013}}} & {\small{}43~487}&{\small{}159~0(8)} & {\small{} \cite{Puchalski2013}} & {\small{}-0}&{\small{}000~4} & {\small{}57}&{\small{}1} & {\small{}Experiment}\tabularnewline
{\small{}$^{9}$Be} & {\small{}Be$^{+}\left(2^{2}S\right)\leftarrow\text{Be}\left(2^{1}S\right)$} & {\small{}75~192}&{\small{}64(6)} & \multicolumn{2}{c}{} & {\small{}75~192}&{\small{}699(7)} & {\small{} \cite{Puchalski2013b}} & {\small{}0}&{\small{}059} & {\small{}0}&{\small{}98} & {\small{}Theory}\tabularnewline
{\small{}$^{11}$B} & {\small{}B$^{+}\left(2^{1}S\right)\leftarrow\text{B}\left(2^{2}P\right)$} & {\small{}66~928}&{\small{}036(22)} & \multicolumn{2}{c}{} & {\small{}66~927}&{\small{}91(21)} & {\small{}\cite{Puchalski2015}} & {\small{}-0}&{\small{}126} & {\small{}5}&{\small{}7} & {\small{}Experiment}\tabularnewline
{\small{}$^{12}$C} & {\small{}C$^{+}\left(2^{2}P\right)\leftarrow\text{C}\left(2^{3}P\right)$} & {\small{}90~833}&{\small{}008(009)} & \multicolumn{2}{c}{} & \textcolor{red}{\small{}90~832}&\textcolor{red}{\small{}014} & {\small{}-} & {\small{}0}&{\small{}994} & {\small{}110}&{\small{}4} & {\small{}Experiment}\tabularnewline[2mm]
\hline 
\end{tabular*}
\end{table*}
\par\end{center}

\begin{center}
\begin{table*}
\caption{\label{tab:excitationEnergies-2}The most precisely calculated electronic
excitation energies for the first 6 atoms, compared to the best known
experimental measurements to date. The last column indicates that
if aiming for the best precision, an experimental measurement is still
the best way to obtain the energy for most atoms, but for Be, the
energy has been obtained more precisely $\textit{in silico}$ than
in any experiment to date.}

\begin{tabular*}{1\textwidth}{@{\extracolsep{\fill}}ccr@{\extracolsep{0pt}.}lr@{\extracolsep{0pt}.}lr@{\extracolsep{0pt}.}l}
\hline 
\noalign{\vskip2mm}
 &  & \multicolumn{2}{c}{} & \multicolumn{2}{c}{} & \multicolumn{2}{c}{}\tabularnewline
 &  & \multicolumn{2}{c}{} & \multicolumn{2}{c}{} & \multicolumn{2}{c}{}\tabularnewline[2mm]
\hline 
\hline 
\noalign{\vskip2mm}
$\alpha^{2}$ & $\eta^{0}$ & \multicolumn{2}{c}{} & \textcolor{red}{90~863}&\textcolor{red}{037(800)} & \multicolumn{2}{c}{}\tabularnewline
$\alpha^{4}$ & $\eta^{0}$ & \multicolumn{2}{c}{\multirow{1}{*}{}} & \multirow{2}{*}{-30&503(050)} & \multicolumn{2}{c}{}\tabularnewline
$\alpha^{5}$ & $\eta^{0}$ & \multicolumn{2}{c}{} & \multirow{2}{*}{-30&503(050)} & \multicolumn{2}{c}{}\tabularnewline[2mm]
\noalign{\vskip2mm}
$\alpha^{2}$ & $\eta^{1}$ & \multicolumn{2}{c}{} & -0&500(000) & \multicolumn{2}{c}{}\tabularnewline
$\alpha^{4}$ & $\eta^{1}$ & \multicolumn{2}{c}{} & 0&000~0(000~4) & \multicolumn{2}{c}{}\tabularnewline
$\alpha^{5}$ & $\eta^{1}$ & \multicolumn{2}{c}{} & 0&000~0(000~1) & \multicolumn{2}{c}{}\tabularnewline
$\alpha^{6}$ & $\eta^{1}$ & \multicolumn{2}{c}{} & 0&000~00(000~02) & \multicolumn{2}{c}{}\tabularnewline[2mm]
\noalign{\vskip2mm}
$\alpha^{2}$ & $\eta^{2}$ & \multicolumn{2}{c}{} & 0&000~0(000~6) & \multicolumn{2}{c}{}\tabularnewline[2mm]
\noalign{\vskip2mm}
\multicolumn{2}{l}{Total (present)} & \multicolumn{2}{c}{} & \textcolor{blue}{90~832}&\textcolor{blue}{034(802)} & \multicolumn{2}{c}{}\tabularnewline[2mm]
\multicolumn{2}{l}{Experiment} & \multicolumn{2}{c}{} & 90~833&11(10) & \multicolumn{2}{c}{}\tabularnewline
\multicolumn{2}{l}{Theory (2010)} & \multicolumn{2}{c}{} & \multicolumn{2}{c}{} & \multicolumn{2}{c}{}\tabularnewline
\multicolumn{2}{c}{Theory (2017)} & \multicolumn{2}{c}{} & \multicolumn{2}{c}{} & \multicolumn{2}{c}{}\tabularnewline
\multicolumn{2}{c}{Theory (2015)} & \multicolumn{2}{c}{} & 90~774& & -59&\tabularnewline[2mm]
\hline 
\end{tabular*}
\end{table*}
\par\end{center}

\begin{center}
\begin{table}
\caption{\label{tab:x2cAndDBOC}Basis set and correlation convergence of the
X2C and DBOC corrections to the C$\left(^{3}P\right)\rightarrow$$\text{C}^{+}\left(^{2}P\right)$
ionization energy in cm$^{-1}$. The difference between CCSDT and
FCI for the smallest two basis sets is denoted by $\Delta_{{\rm correlaton}}$
and the values at aCV3Z are added to the CCSDT energies for all larger
basis sets (resulting in FCI estimates presented in \emph{italic}
font). The numbers in parentheses for the CBS (complete basis set)
estimates denote our assigned uncertainties in the last digits presented,
which we believe to be very conservative. aCVXZ is short for aug-cc-pCVXZ-uncontracted.}

\begin{tabular*}{0.5\textwidth}{@{\extracolsep{\fill}}ccccccc}
\hline 
\noalign{\vskip2mm}
 & {\footnotesize{}aCV2Z} & {\footnotesize{}aCV3Z} & {\footnotesize{}aCV4Z} & {\footnotesize{}aCV5Z} & {\footnotesize{}aCV6Z} & {\footnotesize{}CBS}\tabularnewline[2mm]
\hline 
\hline 
\noalign{\vskip2mm}
\multicolumn{7}{c}{{\footnotesize{}X2C}}\tabularnewline[2mm]
\hline 
\noalign{\vskip2mm}
{\footnotesize{}CCSDT} & {\footnotesize{}-31.055} & {\footnotesize{}-30.144} & {\footnotesize{}-30.037} & {\footnotesize{}-30.004} & {\footnotesize{}-30.001} & {\footnotesize{}-29.999(003)}\tabularnewline
{\footnotesize{}FCI} & {\footnotesize{}-31.074} & {\footnotesize{}-30.168} & \emph{\footnotesize{}-30.060}{\footnotesize{}} & \emph{\footnotesize{}-30.028}{\footnotesize{}} & \emph{\footnotesize{}-30.025} & \textcolor{red}{\emph{\footnotesize{}-30.023(050)}}\tabularnewline
{\footnotesize{}$\Delta_{{\rm correlation}}$} & {\footnotesize{}-0.019} & {\footnotesize{}-0.024} & \emph{\footnotesize{}-0.024} & \emph{\footnotesize{}-0.024} & \emph{\footnotesize{}-0.024} & \emph{\footnotesize{}-0.024(050)}\tabularnewline[2mm]
\hline 
\noalign{\vskip2mm}
\multicolumn{7}{c}{{\footnotesize{}DBOC}}\tabularnewline[2mm]
\hline 
\noalign{\vskip2mm}
{\footnotesize{}CCSDT} & {\footnotesize{}-0.5} & {\footnotesize{}-0.5} & {\footnotesize{}-0.5} & {\footnotesize{}-0.5} & {\footnotesize{}-0.5} & {\footnotesize{}-0.5}\tabularnewline
{\footnotesize{}FCI} & {\footnotesize{}-0.5} & {\footnotesize{}-0.5} & \emph{\footnotesize{}-0.5} & \emph{\footnotesize{}-0.5} & \emph{\footnotesize{}-0.5} & \textcolor{red}{\emph{\footnotesize{}-0.5}}\tabularnewline
{\footnotesize{}$\Delta_{{\rm correlation}}$} & {\footnotesize{}0.0} & {\footnotesize{}0.0} & \emph{\footnotesize{}0.0} & \emph{\footnotesize{}0.0} & \emph{\footnotesize{}0.0} & \emph{\footnotesize{}0.0}\tabularnewline[2mm]
\hline 
\end{tabular*}
\end{table}
\par\end{center}

\begin{center}
\begin{table}
\caption{\label{tab:tightFunctions} Tight exponents for aug-cc-pCV6Z and aug-cc-pCV7Z.}

\begin{tabular*}{0.5\textwidth}{@{\extracolsep{\fill}}cr@{\extracolsep{0pt}.}lr@{\extracolsep{0pt}.}l}
\hline 
\noalign{\vskip2mm}
 & \multicolumn{2}{c}{aug-cc-pCV7Z} & \multicolumn{2}{c}{aug-cc-pCV8Z}\tabularnewline[2mm]
\hline 
\hline 
\noalign{\vskip2mm}
$s$-type & 276&12 & 365&52\tabularnewline
 & 158&30 & 232&20\tabularnewline
 & 90&75 & 147&51\tabularnewline
 & 52&026 & 93&707\tabularnewline
 & 29&826 & 59&529\tabularnewline
 & 17&099 & 37&817\tabularnewline
 & \multicolumn{2}{c}{} & 24&024\tabularnewline[2mm]
$p$-type & 299&20 & 372&10\tabularnewline
 & 149&46 & 198&95\tabularnewline
 & 74&657 & 106&37\tabularnewline
 & 37&293 & 56&868\tabularnewline
 & 18&629 & 30&405\tabularnewline
 & 9&3054 & 16&256\tabularnewline
 & \multicolumn{2}{c}{} & 8&6911\tabularnewline[2mm]
$d$-type & 255&63 & 337&98\tabularnewline
 & 111&57 & 191&20\tabularnewline
 & 55&17 & 108&17\tabularnewline
 & 27&282 & 61&193\tabularnewline
 & 13&491 & 34&618\tabularnewline
 & \multicolumn{2}{c}{} & 19&584\tabularnewline[2mm]
$f$-type & 132&26 & 207&91\tabularnewline
 & 56&375 & 103&16\tabularnewline
 & 24&030 & 51&185\tabularnewline
 & 10&243 & 25&397\tabularnewline
 & \multicolumn{2}{c}{} & 12&601\tabularnewline[2mm]
$g$-type & 94&488 & 155&52\tabularnewline
 & 36&762 & 72&601\tabularnewline
 & 14&302 & 33&891\tabularnewline
 & \multicolumn{2}{c}{} & 15&821\tabularnewline[2mm]
$h$-type & 66&227 & 100&23\tabularnewline
 & 22&68 & 39&552\tabularnewline
 & \multicolumn{2}{c}{} & 15&608\tabularnewline[2mm]
$i$-type & 48&32 & 78&691\tabularnewline
 & \multicolumn{2}{c}{} & 28&941\tabularnewline[2mm]
$k$-type & \multicolumn{2}{c}{} & \multicolumn{2}{c}{}\tabularnewline[2mm]
\hline 
\end{tabular*}
\end{table}
\par\end{center}

\section{Methodology}

Our calculations in this paper can be divided into four stages: (A)
We develop more accurate basis sets than previously available for
carbon, (B) we calculate the 1- and 2-electron integrals in these
basis sets, (C) we attempt to solve the NR,CPN Schroedinger equation
at the FCI level, and (D) we estimate the size of corrections due
to special relativity and due to the atom having an unclamed nucleus.

\subsection{Optimization of `tight function' exponents for the aug-cc-pCV7Z and
aug-cc-pCV8Z basis sets.}

The largest basis sets known for C prior to this work were the (aug-cc-pV$X$Z,
$X=$7,8,9) sets used by Feller earlier this year. These basis sets
did not contain `tight' exponent functions for capturing the effects
of the correlation between the core $(1s^{2},2s^{2})$ electrons and
the valence electrons $(2p^{2}$). The largest known basis set prior
to this work including the CV (core-valence) correction was the aug-cc-pCV6Z
set. In this work we start by optimizing the `tight' exponents for
the CV correction to Feller's 2016 aug-cc-pV7Z and aug-cc-pV8Z basis
sets, yielding the first aug-cc-pCV7Z basis set for carbon, and the
first aug-cc-pCV8Z basis set known.

Optimizing exponents for such large basis sets can be quite a challenging
task. We employ two different approaches for the optimization of the
`tight' functions in our CV corrections. \textcolor{red}{{[}David's
approach, and Jacek's approach{]}.}

\subsection{Calculation of 1- and 2-electron integrals including $k$- and $l$-
functions}

The calculation of the 1- and 2-electron integrals for (aug)-cc-p(C)V$X$Z
basis sets with $X\ge7$ is not possible with most quantum chemistry
packages, since  very few software packages can write 1- and 2-electron
integrals in ${\tt FCIDUMP}$ format, while also supporting $k$-
and $l$- functions, but $k$-functions appear in $X=7$ basis sets
and $l$-functions appear when $X=8$. We have used ${\tt MOLCAS}$
to write the ${\tt FCIDUMP}$ files, and we used a CAS(6,6) for the
$^{1}$D state.... The commercial verison of ${\tt MOLCAS}$ had a
memory leak when printing the 56~GB ${\tt FCIDUMP}$ file for aug-cc-pCV8Z,
so we had to make some changes to ${\tt MOLCAS}$ ...

\subsection{Calculation of non-relativistic energies including all possible levels
of excitation (with FCIQMC)}

We expand the wavefunction of the relavent electornic state as a sum
of all possible Slater determinants (full configuration interaction),
and the coefficients are determined by the number of walkers standing
on each determinant after a Monte Carlo sampling using the NR-CPN
Hamiltonian. The method was introduced in \cite{Booth2009a}, and
we use the initiator method first described in \cite{Cleland2010},
and the semi-stochastic method as described in \cite{Blunt2015}.
The calculations are performed using Version X.X of the software ${\tt NECI}$.

Within a given Hamiltonian (in this case the NR-CPN Hamiltonian) and
basis set, there are three sources of error in the FCIQMC energy calculations: 
\begin{enumerate}
\item Trial wavefunction error ($\Delta E_{{\rm trial}}$), which approaches
zero in the limit where the number of determinants used in the trial
wavefunction approaches the number of determinants in the FCIQMC wavefunction; 
\item Initiator error ($\Delta E_{{\rm initiator}}$), which approaches
zero in the limit where the number of walkers $N_{{\rm walkers}}$
gets sufficiently large; and 
\item Stochastic error ($\Delta E_{{\rm stoch}}$), which for a given number
of walkers is estimated as the square root of the unbiased variance
among different estimates $E_{i}$ of the energy from their mean $\bar{E}$
after different numbers $N$ of Monte Carlo iterations after the walkers
have reached equilibrium: $\Delta E_{{\rm stochastic}}\approx\sqrt{\frac{\sum_{i=1}^{N}\left(E_{i}-\bar{E}\right)^{2}}{N-1}}=\mathcal{O}\left(\nicefrac{1}{\sqrt{N}}\right)$.
\end{enumerate}
Our goal was to obtian all energies to a precision of $\pm\epsilon$
where $\epsilon\le1\mu E_{{\rm Hartree}}\approx0.2$~cm$^{-1}$ (within
the aug-cc-pCV7Z basis set used). Therefore, every calculation was
run for enough iterations $N$ such that $\Delta E_{{\rm stoch}}$
was smaller than $1\mu E_{{\rm Hartree}}$. To ensure that $\Delta E_{{\rm initiator}}$
can be neglected, we used a sufficiently large value of $N_{{\rm walkers}}$
for every energy calculation, so that the energy difference between
using $N_{{\rm walkers}}$ and $\frac{1}{2}N_{{\rm walkers}}$ was
at least an order of magnitude smaller than $\Delta E_{{\rm stochastic}}$.
Likewise, to ensure that $\Delta E_{{\rm estimator}}$ can be neglected,
we used a sufficiently large number of determinants for every energy
calculation, such that $\Delta E_{{\rm estimator}}$ would be at least
an order of magnitude smaller than $\Delta E_{{\rm stochastic}}$.


\subsection{Estimation of relativistic corrections and finite nuclear mass corrections}

\begin{align}
E_{\textrm{corr,CBS}} & =E_{\textrm{corr},X}-\nicefrac{A}{X^{3}},~\textrm{and}.\\
E_{\textrm{corr,CBS}} & =E_{\textrm{corr},X}-\nicefrac{A}{\left(X+\nicefrac{1}{2}\right)}^{2}
\end{align}


\section{Results}
\begin{center}
\begin{table*}
\caption{\label{tab:excitationEnergies-1}FCIQMC}

\begin{tabular*}{1\textwidth}{@{\extracolsep{\fill}}lclclr@{\extracolsep{0pt}.}l}
\hline 
\noalign{\vskip2mm}
 &  &  & \multicolumn{2}{c}{$E=\frac{\left\langle \psi_{N_{{\rm trial}}}\left|H_{{\rm NR-CPN}}\right|\psi_{{\rm FCIQMC}}\right\rangle }{\left\langle \psi_{{\rm N_{{\rm trial}}}}\left.\right|\psi_{{\rm FCIQMC}}\right\rangle }$} & \multicolumn{2}{c}{$E_{{\rm excitation}}$}\tabularnewline[2mm]
\noalign{\vskip2mm}
$N_{{\rm walkers}}$ & $N_{{\rm trial}}$  & Uncertainty & $2^{3}P$~{[}$E_{{\rm hartree}}]$ & $2^{2}P$~{[}$E_{{\rm hartree}}]$ & \multicolumn{2}{c}{$2^{3}P\rightarrow2^{2}P$~{[}cm$^{-1}${]}}\tabularnewline[2mm]
\hline 
\hline 
\noalign{\vskip2mm}
\multicolumn{7}{c}{aug-cc-pCV7Z}\tabularnewline[2mm]
\hline 
\noalign{\vskip2mm}
$64\times10^{6}$ & 1 &  & -37.844~251~5(15) ~~~~~~~ & -37.430~345~0(03) & 90~841&955\tabularnewline
 & 1000 &  & -37.844~251~5(08) ~~~~~~~ & -37.430~345~0(01) & 90~841&977\tabularnewline
 &  & $\Delta E_{{\rm trial}}$ & $<\,$0.000~000~1~~~~~~~ & $<\,$0.000~000~1~~~~~~~ & $<\,$0&02\tabularnewline
$128\times10^{6}$ & 1 &  & -37.844~251~5(0?) & -37.430~345~0(02) & 90~841&977\tabularnewline
 & 1000 &  & -37.844~251~5(05) & -37.430~345~1(01) & 90841&955\tabularnewline
 &  & $\Delta E_{{\rm trial}}$ & $<\,$0.000~000~1~~~~~~~ & $<\,$0.000~000~1~~~~~~~ & $<\,$0&02\tabularnewline
 &  & $\Delta E_{{\rm initiator}}$ & $<\,$0.000~000~1~~~~~~~ & $<\,$0.000~000~1~~~~~~~ & $<\,$0&02\tabularnewline[2mm]
\hline 
\hline 
\noalign{\vskip2mm}
\multicolumn{7}{c}{aug-cc-pCV8Z}\tabularnewline[2mm]
\hline 
\noalign{\vskip2mm}
$64\times10^{6}$ & 1 &  & -37.844~355~5(10) ~~~~~~~ & -37.430~412~3(05) & 90~850&031\tabularnewline
 & 1000 &  & -37.844~355~5(10) ~~~~~~~ & -37.430~412~3(05) & 90~850&031\tabularnewline
 &  & $\Delta E_{{\rm trial}}$ & $<\,$0.000~000~1~~~~~~~ & $<\,$0.000~000~1~~~~~~~ & $<\,$0&001\tabularnewline[2mm]
\hline 
\noalign{\vskip2mm}
$128\times10^{6}$ & 1 &  & -37.844~355~5(08) & -37.430~412~5(05) & 90~849&987\tabularnewline
 & 1000 &  & -37.844~355~5(08) & -37.430~412~5(05) & 90~849&987\tabularnewline
 &  & $\Delta E_{{\rm trial}}$ & $<\,$0.000~000~1~~~~~~~ & $<\,$0.000~000~1~~~~~~~ & $<\,$0&001\tabularnewline
 &  & $\Delta E_{{\rm initiator}}$ & $<\,$0.000~000~1~~~~~~~ & $<\,$0.000~000~2~~~~~~~ & $<\,$0&05\tabularnewline[2mm]
\hline 
\hline 
\noalign{\vskip2mm}
\multicolumn{7}{c}{CBS Extrapolation }\tabularnewline[2mm]
\hline 
\noalign{\vskip2mm}
$1/X^{3.5}$ &  &  & -37.844~528~6 & -37.430~523~6 & 90~863&604\tabularnewline
$1/\left(X+\nicefrac{1}{2}\right)^{4}$ &  &  & -37.844~514~2 & -37.430~514~3 & 90~862&471\tabularnewline
Mean &  &  & -37.844~521~4 & -37.843~051~9 & 90~863&037\tabularnewline
 &  & $\Delta_{\text{basis set}}$ &  &  & 0&800\tabularnewline[2mm]
\hline 
\noalign{\vskip2mm}
\multicolumn{4}{l}{Non-relativistic, clamped nucleus (FCIQMC)} &  & \textcolor{red}{90~863}&\textcolor{red}{037(800)}\tabularnewline
\multicolumn{4}{l}{X2C Relativistic correction} &  & -30&023(050)\tabularnewline
\multicolumn{4}{l}{Breit + QED (from )} &  & -0&480\tabularnewline
\multicolumn{4}{l}{DBOC unclamped nucleus correction} &  & -0&500(010)\tabularnewline[2mm]
\hline 
\noalign{\vskip2mm}
Theory & (Present Work) & \multicolumn{3}{l}{FCIQMC + X2C + Breit + QED + DBOC} & \textcolor{blue}{90~832}&\textcolor{blue}{034(802)}\tabularnewline
Experiment & (1990) & \multicolumn{3}{l}{(with spin-orbit raising)} & 90~833&11(10)\tabularnewline
Experiment &  & \multicolumn{2}{l}{According to Feller} &  & 90~831&89\tabularnewline
Theory & (2010) & \multicolumn{3}{l}{CCSD-F12 + HLC + MV + Darwin + Breit + QED + DBOC} & 90~826&09\tabularnewline
Theory & (2017) & \multicolumn{3}{l}{R/UCCSD(T) + HLC + DKH + Breit + QED + DBOC} & 90~827&50\tabularnewline
Theory & (2015) & \multicolumn{2}{l}{FN-DMC} &  & 90~774&\tabularnewline[2mm]
\hline 
QED &  &  &  &  & +1&076~$\pm0.806$\tabularnewline[2mm]
\end{tabular*}

\end{table*}
\par\end{center}

\section{Dicsussion}

\selectlanguage{british}%
\bibliographystyle{apsrev4-1}
\bibliography{\string"/home/nike/pCloud Sync/library\string",\string"/Users/leroy/Dropbox (Personal)/NSD/Papers/library\string",C:/Users/Nike/Desktop/Dropbox/NSD/Papers/library,C:/Users/Richard/Documents/Bibtex/library,apssamp}
\selectlanguage{english}%

\end{document}
