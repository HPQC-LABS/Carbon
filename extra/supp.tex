%% LyX 2.2.2 created this file.  For more info, see http://www.lyx.org/.
%% Do not edit unless you really know what you are doing.
\documentclass[british,english,showpacs,preprintnumbers,amsmath,amssymb,aps,notitlepage]{revtex4-1}
\usepackage{lmodern}
\renewcommand{\sfdefault}{lmss}
\renewcommand{\ttdefault}{lmtt}
\usepackage[T1]{fontenc}
\usepackage[latin9]{inputenc}
\usepackage{geometry}
\geometry{verbose,tmargin=0.5cm,bmargin=1.8cm,lmargin=2cm,rmargin=2cm}
\setcounter{secnumdepth}{3}
\usepackage{color}
\usepackage{float}
\usepackage{amstext}
\usepackage{graphicx}

\makeatletter

%%%%%%%%%%%%%%%%%%%%%%%%%%%%%% LyX specific LaTeX commands.
%% Because html converters don't know tabularnewline
\providecommand{\tabularnewline}{\\}

%%%%%%%%%%%%%%%%%%%%%%%%%%%%%% User specified LaTeX commands.
\usepackage{amsfonts}
%\usepackage{palatino}

\makeatother

\usepackage{babel}
\begin{document}

\title{\textcolor{black}{\Large{}Supplemental Material}}
\maketitle

\section{Spin-orbit lowering}

{\small{}In the caption to Table III of the main paper we said that
details about the derivation of the spin-orbit lowering value we used
($-12.702$ cm$^{-1}$) would be presented in the Supplemental Material.
All numbers in the weighted averaging of our Table below come from
\cite{Chang1998}.}{\small \par}
\begin{center}
\begin{table}[H]
\caption{Details about the spin-orbit lowering value.}

\begin{centering}
\begin{tabular*}{0.7\textwidth}{@{\extracolsep{\fill}}clr@{\extracolsep{0pt}.}l}
\hline 
\noalign{\vskip2mm}
{\footnotesize{}System} & {\footnotesize{}Weighted Averaging Calculation} & \multicolumn{2}{c}{{\footnotesize{}Result {[}cm$^{-1}${]}}}\tabularnewline[2mm]
\hline 
\noalign{\vskip2mm}
{\small{}C$\left(^{3}P\right)$} & {\small{}$\left(\text{0.00\ensuremath{\times}1 + 16.40\ensuremath{\times}3 + 43.40\ensuremath{\times}5}\right)/9$ } & {\small{}29}&{\small{}578}\tabularnewline
{\small{}C$^{+}$$\left(^{2}P\right)$} & {\small{}$\left(\text{0.00\ensuremath{\times}2 + 63.42\ensuremath{\times}4}\right)/6$} & {\small{}42}&{\small{}280}\tabularnewline
{\small{}Difference} &  & {\small{}-12}&{\small{}702}\tabularnewline[2mm]
\hline 
\end{tabular*}
\par\end{centering}
\end{table}
\par\end{center}

{\small{}After the completion of this work, Haris and Kramida published
more accurate results which lead to the following \cite{Haris2017}:}{\small \par}
\begin{center}
\begin{table}[H]
\caption{Details about the spin-orbit lowering value of {\small{}\cite{Haris2017}}.}

\centering{}%
\begin{tabular*}{0.7\textwidth}{@{\extracolsep{\fill}}clr@{\extracolsep{0pt}.}l}
\hline 
\noalign{\vskip2mm}
{\footnotesize{}System} & {\footnotesize{}Weighted Averaging Calculation} & \multicolumn{2}{c}{{\footnotesize{}Result {[}cm$^{-1}${]}}}\tabularnewline[2mm]
\hline 
\noalign{\vskip2mm}
{\small{}C$\left(^{3}P\right)$} & {\small{}$\left(\text{0.00\ensuremath{\times}1 + 16.4167130\ensuremath{\times}3 + 43.4134567\ensuremath{\times}5}\right)/9$ } & {\small{}29}&{\small{}590~825}\tabularnewline
{\small{}C$^{+}$$\left(^{2}P\right)$} & {\small{}$\left(\text{0.00\ensuremath{\times}2 + 63.395\ensuremath{\times}4}\right)/6$} & {\small{}42}&{\small{}263~333}\tabularnewline
{\small{}Difference} &  & {\small{}-12}&{\small{}672~508}\tabularnewline[2mm]
\hline 
\end{tabular*}
\end{table}
\par\end{center}

\section{Experimental energies}
\begin{center}
\begin{table}[H]
\caption{Details about the experimental energies (in cm$^{-1}$). The line
labelled by {[}{*}{]} is the Chang \& Geller value but using 90883.854(15)-63.39509(2)
as described in {\small{}\cite{Haris2017} but never calculated/reported
there.}}

\centering{}%
\begin{tabular*}{0.7\textwidth}{@{\extracolsep{\fill}}ccclr@{\extracolsep{0pt}.}lr@{\extracolsep{0pt}.}l}
\hline 
\noalign{\vskip2mm}
Author(s) & Year & Reference & {\footnotesize{}$^{3}P_{0}\rightarrow^{2}P_{1/2}$} & \multicolumn{2}{c}{{\footnotesize{}SO Lowering}} & \multicolumn{2}{c}{Difference in COGs}\tabularnewline[2mm]
\hline 
\noalign{\vskip2mm}
{\small{}Johansson} & 1966 & \cite{Johansson1966} & 90820.420(100) & \multicolumn{2}{c}{-} & \multicolumn{2}{c}{-}\tabularnewline
{\small{}Chang \& Geller} & 1998 & {\small{}\cite{Chang1998}} & 90820.469(015) & 12&702 & 90833&171\tabularnewline
{[}{*}{]} & {[}{*}{]} & {[}{*}{]} & 90820.458(015) & \multicolumn{2}{c}{-} & \multicolumn{2}{c}{-}\tabularnewline
{\small{}Haris and Kramida} & 2017 & {\small{}\cite{Haris2017}} & 90820.310(030) & \multicolumn{2}{c}{-} & \multicolumn{2}{c}{-}\tabularnewline
{\small{}Haris and Kramida} & 2017 & {\small{}\cite{Haris2017}} & 90820.348(009) & 12&672508 & 90833&021\tabularnewline[2mm]
\hline 
\end{tabular*}
\end{table}
\par\end{center}

\section{Estimated uncertainties from the FCIQMC calculations}

{\small{}In Section 1C of the main paper, we said that we would present
details about how we arrived at our FCIQMC energies and uncertainties
in them, for the aug-cc-pCV7Z and aug-cc-pCV8Z basis sets. Our best
FCIQMC calculations (largest number of walkers, and largest number
of configuration interaction Slater determinents included in $|\psi_{{\rm trial}}\rangle$)
for the neutral atom and cation, for each basis set, are in red font
in Table \ref{tab:excitationEnergies-1-1}, and the stochastic uncertainties
on their last two digits are in parentheses. }{\small \par}
\begin{center}
\begin{table*}
\caption{\label{tab:excitationEnergies-1-1}Details about the FCIQMC calculations.}

\begin{tabular*}{1\textwidth}{@{\extracolsep{\fill}}lclr@{\extracolsep{0pt}.}lr@{\extracolsep{0pt}.}lr@{\extracolsep{0pt}.}l}
\hline 
\noalign{\vskip2mm}
 &  &  & \multicolumn{4}{c}{$E=\frac{\left\langle \psi_{N_{{\rm trial}}}\left|H_{{\rm NR-CPN}}\right|\psi_{{\rm FCIQMC}}\right\rangle }{\left\langle \psi_{N_{{\rm trial}}}\left.\right|\psi_{{\rm FCIQMC}}\right\rangle }$} & \multicolumn{2}{c}{$E_{{\rm ionization}}$}\tabularnewline[2mm]
\noalign{\vskip2mm}
$N_{{\rm walkers}}$ & $N_{{\rm trial}}$  & Uncertainty & \multicolumn{2}{c}{C$\left(2^{3}P\right)$~{[}$E_{{\rm Hartree}}]$} & \multicolumn{2}{c}{$\text{C}^{+}\left(2^{2}P\right)$~{[}$E_{{\rm Hartree}}]$} & \multicolumn{2}{c}{{[}cm$^{-1}${]}}\tabularnewline[2mm]
\hline 
\hline 
\noalign{\vskip2mm}
\multicolumn{9}{c}{aug-cc-pCV7Z}\tabularnewline[2mm]
\hline 
\noalign{\vskip2mm}
$64\times10^{6}$ & 1 &  & -37&844~251~5(9) ~~~~~~~ & -37&430~345~0(3) & 90~841&955\tabularnewline
 & 1000 &  & -37&844~251~5(8) ~~~~~~~ & -37&430~345~0(1) & 90~841&977\tabularnewline
 &  & $\Delta E_{{\rm trial}}$ & $<\,$0&000~000~1~~~~~~~ & $<\,$0&000~000~1~~~~~~~ & $<\,$0&02\tabularnewline
$128\times10^{6}$ & 1 &  & -37&844~251~5(8) & -37&430~345~0(2) & 90~841&977\tabularnewline
 & 1000 &  & \textcolor{red}{-37}&\textcolor{red}{844~251~5(5)} & \textcolor{red}{-37}&\textcolor{red}{430~345~1(1)} & \textcolor{red}{90~841}&\textcolor{red}{955}\tabularnewline
 &  & $\Delta E_{{\rm trial}}$ & $<\,$0&000~000~1~~~~~~~ & $<\,$0&000~000~1~~~~~~~ & $<\,$0&02\tabularnewline
 &  & $\Delta E_{{\rm initiator}}$ & $<\,$0&000~000~1~~~~~~~ & $<\,$0&000~000~1~~~~~~~ & $<\,$0&02\tabularnewline[2mm]
\hline 
\hline 
\noalign{\vskip2mm}
\multicolumn{9}{c}{aug-cc-pCV8Z}\tabularnewline[2mm]
\hline 
\noalign{\vskip2mm}
$64\times10^{6}$ & 1 &  & -37&844~355~5(9) ~~~~~~~ & -37&430~412~3(5) & 90~850&031\tabularnewline
 & 1000 &  & -37&844~355~5(9) ~~~~~~~ & -37&430~412~3(5) & 90~850&031\tabularnewline
 &  & $\Delta E_{{\rm trial}}$ & $<\,$0&000~000~1~~~~~~~ & $<\,$0&000~000~1~~~~~~~ & $<\,$0&02\tabularnewline[2mm]
\hline 
\noalign{\vskip2mm}
$128\times10^{6}$ & 1 &  & -37&844~355~5(8) & -37&430~412~5(5) & 90~849&987\tabularnewline
 & 1000 &  & \textcolor{red}{-37}&\textcolor{red}{844~355~5(8)} & \textcolor{red}{-37}&\textcolor{red}{430~412~5(5)} & \textcolor{red}{90~849}&\textcolor{red}{987}\tabularnewline
 &  & $\Delta E_{{\rm trial}}$ & $<\,$0&000~000~1~~~~~~~ & $<\,$0&000~000~1~~~~~~~ & $<\,$0&02\tabularnewline
 &  & $\Delta E_{{\rm initiator}}$ & $<\,$0&000~000~1~~~~~~~ & $<\,$0&000~000~2~~~~~~~ & $<\,$0&05\tabularnewline[2mm]
\hline 
\end{tabular*}
\end{table*}
\par\end{center}

{\small{}In all cases displayed in Table \ref{tab:excitationEnergies-1-1},
the differene between projecting onto only the Hartree-Fock determinant,
or projecting onto a trial wavefunction with the leading 1000 Slater
determinants is at most 0.000~000~1 $E_{{\rm Hartree}}$. Therefore
we assign $\Delta E_{{\rm trial}}$ to be no larger than 0.000~000~1
$E_{{\rm Hartree}}$ or 0.02~cm$^{-1}$. }{\small \par}

{\small{}In all cases displayed in Table \ref{tab:excitationEnergies-1-1},
the difference between using 64 million walkers and 128 million walkers
was no larger than 0.000~000~2 $E_{{\rm Hartree}}$, and since the
initiator approximation vanishes in the limit of large walker number,
we assign $\Delta E_{{\rm initiator}}$ to be no larger than 0.000~000~2
$E_{{\rm Hartree}}$ or 0.05~cm$^{-1}$. For aug-cc-pCV7Z, the difference
between 64 million and 128 million walkers was no larger than 0.000~000~1
$E_{{\rm Hartree}}$, so $\Delta E_{{\rm initiator}}$ was assigned
to be no larger than 0.02~cm$^{-1}$.}{\small \par}

{\small{}The numbers in parentheses, which denote the stochastic flucuations
in each number of Table \ref{tab:excitationEnergies-1-1}, were smaller
than $\Delta E_{{\rm initiator}}$ and $\Delta E_{{\rm trial}}$ in
all cases. So the final energy ionization energies have uncertainties
dominated by initiator error. To be extra safe, we further increased
our uncertainty estimates based on the formula for adding two independent
uncertainties:}{\small \par}

{\small{}
\begin{equation}
\sqrt{\Delta_{{\rm trial}}^{2}+\Delta_{{\rm initiator}}^{2}},
\end{equation}
and this increases the ionization energy uncertainties to 0.028 for
aug-cc-pCV7Z and 0.054 for aug-cc-pCV8Z. Our final energies are therefore
reported in Table IV of the main paper to be }\textcolor{red}{\small{}90841.955(028)
}{\small{}for aug-cc-pCV7Z and }\textcolor{red}{\small{}90~849.987(054)
}{\small{}for aug-cc-pCV7Z and if there are any further doubts about
the accuracy of the FCIQMC energies, we note that both of these uncertainties
are more than an order of magnitude smaller than our final discrepancy
with experiment of 0.872~cm$^{-1}$, which is primarily from neglect
of higher-order QED terms and from basis set incompleteness. }{\small \par}

\newpage{}

\section{Basis set extrapolation}

{\small{}The next table shows, for each basis set size $X$, the values
for the correlation energies ($E_{X})$ which are needed to extrapolate
to the CBS (complete basis set) limit using the fomulas given in Eqs
(1-4). Hartree-Fock energies are labeled ROHF to denote that we used
Restricted Open Shell Hartree-Fock.}{\small \par}
\begin{center}
\begin{table}[H]

\caption{Correlation energies (in $E_{{\rm Hartree}}$) used for the basis
set extrapolations. }

\begin{centering}
\begin{tabular*}{0.55\textwidth}{@{\extracolsep{\fill}}lr@{\extracolsep{0pt}.}lr@{\extracolsep{0pt}.}lr@{\extracolsep{0pt}.}l}
\hline 
\noalign{\vskip2mm}
{\footnotesize{}Basis set} & \multicolumn{2}{c}{{\footnotesize{}ROHF}} & \multicolumn{2}{c}{{\footnotesize{}FCI}} & \multicolumn{2}{c}{$E_{X}$}\tabularnewline[2mm]
\hline 
\hline 
\noalign{\vskip2mm}
\multicolumn{7}{c}{C$\left(2^{3}P\right)$}\tabularnewline[2mm]
\hline 
\noalign{\vskip2mm}
{\small{}aug-cc-pCV}\textcolor{red}{\small{}5}{\small{}Z} & -37&688~648~2 & -37&842~955 & -0&154~267~695~\tabularnewline
{\small{}aug-cc-pCV}\textcolor{red}{\small{}6}{\small{}Z} & -37&688~687~3~~~~~ & -37&843~840~5~~~~~ & -0&155~153~195~\tabularnewline
{\small{}aug-cc-pCV}\textcolor{red}{\small{}7}{\small{}Z} & -37&688~692~7 & -37&844~251~5 & -0&155~558~777~\tabularnewline
{\small{}aug-cc-pCV}\textcolor{red}{\small{}8}{\small{}Z} & -37&688~693~6 & -37&844~355~5 & -0&155~661~899~\tabularnewline[2mm]
\hline 
\noalign{\vskip2mm}
\multicolumn{7}{c}{C$^{+}$$\left(2^{2}P\right)$}\tabularnewline[2mm]
\hline 
\noalign{\vskip2mm}
{\small{}aug-cc-pCV}\textcolor{red}{\small{}5}{\small{}Z} & -37&292~242~5 & -37&429~265~3 & -0&137~022~886~\tabularnewline
{\small{}aug-cc-pCV}\textcolor{red}{\small{}6}{\small{}Z} & -37&292~283~0 & -37&430~004 & -0&137~720~959~\tabularnewline
{\small{}aug-cc-pCV}\textcolor{red}{\small{}7}{\small{}Z} & -37&292~289~3 & -37&430~345~05 & -0&138~062~009~\tabularnewline
{\small{}aug-cc-pCV}\textcolor{red}{\small{}8}{\small{}Z} & -37&292~290~6 & -37&430~412~5 & -0&138~123~149~\tabularnewline[2mm]
\hline 
\end{tabular*}
\par\end{centering}

\end{table}
\par\end{center}

{\small{}Next, for each basis set size $X$, we use Eqs (1) and (3)
with $n=3.5$ to obtain one CBS estimate of the correlation energy;
and Eqs (2) and (4) with $n=4$, to obtain a second CBS estimate of
the correlation energy. These corelation energies are added to the
CBS estimate of the Hartree-Fock energy, which we estimate to be the
aug-cc-pCV8Z Hartree-Fock energy, to get CBS estimates of the total
energies of C and $\text{C}^{+}$. While it might seem from the Table
that the ROHF energies at aug-cc-pCV8Z are still not good estimates
of the CBS values, we note that the energy }\emph{\small{}differences
}{\small{}between the C and C$^{+}$ ROHF energies, }\textbf{\emph{\small{}are
}}{\small{}indeed converged much more than the total energies (a phenomenon
quite common with Gaussian basis sets since the errors in describing
the shape of the electronic wavefunction for C are similar to the
errors for C$^{+}$, which is why our ionization energy agrees with
experiment by more than an order of magnitude better than the diffusion
Monte Carlo result despite diffusion Monte Carlo having lower variational
total energies). Therefore for each $X>5$, we have a CBS estimate
of the ionization energy obtained with one formula (Eqs. 1 and 3 with
$n=3.5$) and a CBS estimate of the ionization energy obtained with
another formula (Eqs. 2 and 4 with $n=4$). For each $X$, these two
esitimates of the ionization energy can be averaged, and the unbiased
variance around the mean can be calculated. This is what we present
in the next table.}
\begin{table}[h]

\caption{Extrapolated ionization energies.}

\centering{}%
\begin{tabular*}{0.55\textwidth}{@{\extracolsep{\fill}}ccc}
\hline 
\noalign{\vskip2mm}
 & {\small{}Ionization Energy {[}cm$^{-1}${]}} & {\small{}Unbiased Variance {[}cm$^{-1}${]}}\tabularnewline[2mm]
\hline 
\noalign{\vskip2mm}
{\small{}CBS(5,6)} & {\small{}90861.9685} & {\small{}$\pm$~1.5682}\tabularnewline
{\small{}CBS(6,7)} & {\small{}90862.8988} & {\small{}$\pm$~1.1407}\tabularnewline
{\small{}CBS(7,8)} & {\small{}90863.0375} & {\small{}$\pm$~0.8004}\tabularnewline[2mm]
\hline 
\end{tabular*}
\end{table}

These three CBS ionization energy estimates have been fitted to an
exponential of the form $A-Be^{-CX}$ and plotted in the Figure below.

The fact that all points lie roughly between 90862~cm$^{-1}$ and
90863~cm$^{-1}$ makes us doubt that the final point 90863.0375(8004)~cm$^{-1}$
is incorrect by more than 1~cm$^{-1}$.
\begin{center}
\begin{figure}[H]
\caption{Fit of the extrapolated ionization energies to the function $A-Be^{-CX}$
.}

\centering{}\includegraphics[width=0.55\textwidth]{\string"/home/nike/pCloud Sync/atomic/C/ionization/CBSextrapolation\string".pdf}
\end{figure}
\par\end{center}

\vspace{-10mm}


\section{X2C and DBOC}

{\small{}In section 1E we claimed that our X2C (exact 2-component)
calculations and DBOC (diagonal Born-Oppenheimer correction) calculations
were converged with respect to the size of the basis set and level
of correlation used. The table below shows that all numbers are converged
on the scale of about 0.05~cm$^{-1}$. }{\small \par}
\begin{center}
\begin{table}[H]
{\footnotesize{}\caption{\label{tab:x2cAndDBOC}Basis set and correlation convergence of the
X2C and DBOC corrections to the C$\left(^{3}P\right)\rightarrow$
$\text{C}^{+}\left(^{2}P\right)$ ionization energy in cm$^{-1}$.
The difference between CCSDT and FCI (for X2C) and CCSDT and CCSDTQ
(for DBOC) for the smallest basis sets is denoted by $\Delta_{{\rm correlaton}}$
and the values at aug-cc-pCV3Z-unc are added to the CCSDT energies
for all larger basis sets for X2C (and the same is done with aug-cc-pCV4Z
for DBOC), resulting in FCI estimates presented in \emph{italic} font.
The numbers in parentheses for the CBS (complete basis set) estimates
denote our assigned uncertainties in the last digits presented. CBS
values and uncertainties are estimated using the difference between
the values obtained at the two largest basis sets used (this uncertainty
is likely conservative for X2C but too lenient for DBOC, based on
how the numbers were changing in smaller basis sets. The X2C calculations
are done with basis sets that were uncontracted. X2C uses ROHF, whereas
DBOC uses UHF.}
}{\footnotesize \par}

{\footnotesize{}}%
\begin{tabular*}{1\textwidth}{@{\extracolsep{\fill}}ccccccc}
\hline 
\noalign{\vskip2mm}
 & {\footnotesize{}aug-cc-pCV}\textcolor{red}{\footnotesize{}2}{\footnotesize{}Z} & {\footnotesize{}aug-cc-pCV}\textcolor{red}{\footnotesize{}3}{\footnotesize{}Z} & {\footnotesize{}aug-cc-pCV}\textcolor{red}{\footnotesize{}4}{\footnotesize{}Z} & {\footnotesize{}aug-cc-pCV}\textcolor{red}{\footnotesize{}5}{\footnotesize{}Z} & {\footnotesize{}aug-cc-pCV}\textcolor{red}{\footnotesize{}6}{\footnotesize{}Z} & {\footnotesize{}CBS}\tabularnewline[2mm]
\hline 
\hline 
\noalign{\vskip2mm}
\multicolumn{7}{c}{{\footnotesize{}X2C}}\tabularnewline[2mm]
\hline 
\noalign{\vskip2mm}
{\footnotesize{}CCSDT} & {\footnotesize{}-31.055} & {\footnotesize{}-30.144} & {\footnotesize{}-30.037} & {\footnotesize{}-30.004} & {\footnotesize{}-30.001} & {\footnotesize{}-29.999(003)}\tabularnewline
{\footnotesize{}FCI} & {\footnotesize{}-31.074} & {\footnotesize{}-30.168} & \emph{\footnotesize{}-30.060}{\footnotesize{}} & \emph{\footnotesize{}-30.028}{\footnotesize{}} & \emph{\footnotesize{}-30.025} & \textcolor{red}{\emph{\footnotesize{}-30.023(050)}}\tabularnewline
{\footnotesize{}$\Delta_{{\rm correlation}}$} & {\footnotesize{}-0.019} & {\footnotesize{}-0.024} & \emph{\footnotesize{}-0.024} & \emph{\footnotesize{}-0.024} & \emph{\footnotesize{}-0.024} & \emph{\footnotesize{}-0.024(050)}\tabularnewline[2mm]
\hline 
\noalign{\vskip2mm}
\multicolumn{7}{c}{{\footnotesize{}DBOC}}\tabularnewline[2mm]
\hline 
\noalign{\vskip2mm}
{\footnotesize{}CCSDT} & {\footnotesize{}0.147} & {\footnotesize{}-0.185} & {\footnotesize{}-0.223} & {\footnotesize{}-0.225} &  & {\footnotesize{}-0.227(002)}\tabularnewline
{\footnotesize{}CCSDTQ} & {\footnotesize{}0.142} & {\footnotesize{}-0.193} & {\footnotesize{}-0.231} & \emph{\footnotesize{}-0.233} &  & \textcolor{red}{\emph{\footnotesize{}-0.235(002)}}\tabularnewline
{\footnotesize{}$\Delta_{{\rm correlation}}$} & {\footnotesize{}-0.007} & {\footnotesize{}-0.008} & {\footnotesize{}-0.008} & \emph{\footnotesize{}-0.008} &  & \emph{\footnotesize{}-0.008(001)}\tabularnewline[2mm]
\hline 
\end{tabular*}{\footnotesize \par}
\end{table}
\par\end{center}

{\small{}The DBOC calculations were done for the $^{12}$C isotopologue.
However, the weighted average of the DBOCs for $^{12}$C and $^{13}$C
is about the same:}{\small \par}
\begin{center}
\begin{table}[H]
{\footnotesize{}\caption{Weighted averaging of DBOC correction in cm$^{-1}$ at UHF-CCSD/aug-cc-pCV2Z
level.}
}{\footnotesize \par}

{\footnotesize{}}%
\begin{tabular*}{1\textwidth}{@{\extracolsep{\fill}}cccc}
\hline 
\noalign{\vskip2mm}
 & {\footnotesize{}DBOC correction} & {\footnotesize{}Abundance} & {\footnotesize{}Product of previous two columns}\tabularnewline[2mm]
\hline 
\noalign{\vskip2mm}
{\footnotesize{}$^{12}$C} & {\footnotesize{}-0.217} & {\footnotesize{}0.9893} & {\footnotesize{}-0.21458587565661874}\tabularnewline
{\footnotesize{}$^{13}$C} & {\footnotesize{}-0.200} & {\footnotesize{}0.0107} & {\footnotesize{}-0.00214195608061821}\tabularnewline
{\footnotesize{}Sum of previous two rows} &  &  & {\footnotesize{}-0.21672783173723695 }\tabularnewline[2mm]
\hline 
\end{tabular*}{\footnotesize \par}
\end{table}
\par\end{center}

\newpage{}

\section{Finite nuclear-size effects}

{\small{}We estimate the size of the point-nucleus approximation at
the CCSD(T)/aug-cc-pCV6Z-unc level. As in the case of the X2C and
DBOC corrections, the correlation and basis set dependence is much
smaller than 1~cm$^{-1}$, and our final result is that the effect
is far smaller than 1 cm$^{-1}$:}{\small \par}
\begin{center}
\begin{table}[h]
\caption{{\small{}Finite nuclear size effects are small.}}

\centering{}%
\begin{tabular*}{0.35\textwidth}{@{\extracolsep{\fill}}lc}
\hline 
\noalign{\vskip2mm}
 & {\small{}Ionization energy}\tabularnewline
Radius of nucleus {[}fm{]} & {\small{}{[}cm$^{-1}${]}}\tabularnewline[2mm]
\hline 
\noalign{\vskip2mm}
0 & {\small{}~~90753.38064}\tabularnewline
2.7 & {\small{}~~90753.38607}\tabularnewline[2mm]
\hline 
\noalign{\vskip2mm}
{\small{}Difference in Energy} & {\small{}~~~~~~~~0.00543 }\tabularnewline[2mm]
\hline 
\end{tabular*}
\end{table}
\par\end{center}


\section{Tight function exponents optimized in this work}

{\footnotesize{}Finally, we provide the tight exponent functions for
our aug-cc-pCV7Z and aug-cc-pCV8Z basis sets in case others wish to
use them for their own applications.}{\footnotesize \par}

\begin{center}
{\scriptsize{}}
\begin{table}[H]
{\scriptsize{}\caption{\label{tab:tightFunctions} Tight exponents optimized in our work.}
}{\scriptsize \par}

{\scriptsize{}}{\scriptsize \par}
\centering{}{\scriptsize{}}%
\begin{tabular*}{0.35\textwidth}{@{\extracolsep{\fill}}ccc}
\hline 
\noalign{\vskip2mm}
 & {\scriptsize{}aug-cc-pCV7Z} & {\scriptsize{}aug-cc-pCV8Z}\tabularnewline[2mm]
\hline 
\hline 
\noalign{\vskip2mm}
{\scriptsize{}$s$-type} & {\scriptsize{}276.1200} & {\scriptsize{}365.5200}\tabularnewline[-1mm]
 & {\scriptsize{}158.3000} & {\scriptsize{}232.2000}\tabularnewline[-1mm]
 & {\scriptsize{}90.7500} & {\scriptsize{}147.5100}\tabularnewline[-1mm]
 & {\scriptsize{}52.0260} & {\scriptsize{}93.7070}\tabularnewline[-1mm]
 & {\scriptsize{}29.8260} & {\scriptsize{}59.5290}\tabularnewline[-1mm]
 & {\scriptsize{}17.0990} & {\scriptsize{}37.8170}\tabularnewline[-1mm]
 &  & {\scriptsize{}24.0240}\tabularnewline
{\scriptsize{}$p$-type} & {\scriptsize{}299.2000} & {\scriptsize{}372.1000}\tabularnewline[-1mm]
 & {\scriptsize{}149.4600} & {\scriptsize{}198.9500}\tabularnewline[-1mm]
 & {\scriptsize{}74.6570} & {\scriptsize{}106.3700}\tabularnewline[-1mm]
 & {\scriptsize{}37.2930} & {\scriptsize{}56.8680}\tabularnewline[-1mm]
 & {\scriptsize{}18.6290} & {\scriptsize{}30.4050}\tabularnewline[-1mm]
 & {\scriptsize{}9.3054} & {\scriptsize{}16.2560}\tabularnewline[-1mm]
 &  & {\scriptsize{}8.6911}\tabularnewline
{\scriptsize{}$d$-type} & {\scriptsize{}255.6300} & {\scriptsize{}337.9800}\tabularnewline[-1mm]
 & {\scriptsize{}111.5700} & {\scriptsize{}191.2000}\tabularnewline[-1mm]
 & {\scriptsize{}55.1700} & {\scriptsize{}108.1700}\tabularnewline[-1mm]
 & {\scriptsize{}27.2820} & {\scriptsize{}61.1930}\tabularnewline[-1mm]
 & {\scriptsize{}13.4910} & {\scriptsize{}34.6180}\tabularnewline[-1mm]
 &  & {\scriptsize{}19.5840}\tabularnewline
{\scriptsize{}$f$-type} & {\scriptsize{}132.2600} & {\scriptsize{}207.9100}\tabularnewline[-1mm]
 & {\scriptsize{}56.3750} & {\scriptsize{}103.1600}\tabularnewline[-1mm]
 & {\scriptsize{}24.0300} & {\scriptsize{}51.1850}\tabularnewline[-1mm]
 & {\scriptsize{}10.2430} & {\scriptsize{}25.3970}\tabularnewline[-1mm]
 &  & {\scriptsize{}12.6010}\tabularnewline
{\scriptsize{}$g$-type} & {\scriptsize{}94.4880} & {\scriptsize{}155.5200}\tabularnewline[-1mm]
 & {\scriptsize{}36.7620} & {\scriptsize{}72.6010}\tabularnewline[-1mm]
 & {\scriptsize{}14.3020} & {\scriptsize{}33.8910}\tabularnewline[-1mm]
 &  & {\scriptsize{}15.8210}\tabularnewline
{\scriptsize{}$h$-type} & {\scriptsize{}66.2270} & {\scriptsize{}100.2300}\tabularnewline[-1mm]
 & {\scriptsize{}22.6800} & {\scriptsize{}39.5520}\tabularnewline[-1mm]
 &  & {\scriptsize{}15.6080}\tabularnewline
{\scriptsize{}$i$-type} & {\scriptsize{}48.3200} & {\scriptsize{}78.6910}\tabularnewline[-1mm]
 &  & {\scriptsize{}28.9410}\tabularnewline
{\scriptsize{}$k$-type} & {\scriptsize{}N/A} & {\scriptsize{}51.2410}\tabularnewline[2mm]
\hline 
\end{tabular*}{\scriptsize \par}
\end{table}
\par\end{center}{\scriptsize \par}

\selectlanguage{british}%
\bibliographystyle{apsrev4-1}
\bibliography{../../../../library,\string"/Users/leroy/Dropbox (Personal)/NSD/Papers/library\string",C:/Users/Nike/Desktop/Dropbox/NSD/Papers/library,C:/Users/Richard/Documents/Bibtex/library,apssamp}
\selectlanguage{english}%

\end{document}
